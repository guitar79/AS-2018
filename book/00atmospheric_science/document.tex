\section{유효 온도}


%(https://eesc.columbia.edu/courses/ees/slides/climate/insolation_yk.gif)

%\includegraphic[width=0.5\textwidth]{00atmospheric_science/images/insolation_yk.png}

\begin{tikzpicture}
\draw (5,2) circle (3cm);
\fill [yellow!20!white] (0,2) ellipse (1cm and 3cm);
\draw (0,2) ellipse (1cm and 3cm);
\text at (5,2) {Incoming \\solar \\flux $S$};
\fill [blue!20!white] (5,2) ellipse (1cm and 3cm);
\draw (5,2) ellipse (1cm and 3cm);
\draw (10,2) ellipse (1cm and 3cm);
\node[draw] (10,2) -- (10,5);
\text at (16,5) {$\pi R^2$};
\end{tikzpicture}


* 지구에서의 태양 상수 : $ S_{\mathrm{E}} = \dfrac{L_{\odot}}{4\pi r_{\mathrm{E}}^{2}}$

* 지구에 받는 태양 복사 에너지 : $ \pi R_{\mathrm{E}}^{2} S_{\mathrm{E}}$

* 행성에서의 태양 상수 : $ S_{\mathrm{P}} = S_{\mathrm{E}}\left(\dfrac{r_{\mathrm{E}}}{r_{\mathrm{P}}}\right)^{2}$

* 행성이 받는 태양복사에너지 : $ \pi R_{\mathrm{P}}^{2} S_{\mathrm{E}}\left(\dfrac{r_{\mathrm{E}}}{r_{\mathrm{P}}}\right)^{2}$

* 알베도($A$)를 고려한 행성의 행성이 받는 일사량 : $ I_{\mathrm{P}}^{\downarrow} = (1-A)\pi R_{\mathrm{P}}^{2} S_{\mathrm{E}}\left(\dfrac{r_{\mathrm{E}}}{r_{\mathrm{P}}}\right)^{2}$

* Stefan-Boltzmann 법칙 : $ I_{\mathrm{P}}^{\uparrow} = 4\pi R_{\mathrm{P}}^{2} \sigma T^{4}$

* 유효 온도 (effective temperature) : $ T_{e} = \sqrt[4]{\dfrac{(1-A) S_{\mathrm{E}}}{4 \sigma}} \sqrt{\dfrac{r_{\mathrm{E}}}{r_{\mathrm{P}}}} $


유효 온도는 행성과 태양과의 거리, 알베도에 의해 결정되며 대기의 구성 성분이나 밀도 등의 물리적 성질과는 무관하다.

그러나 실제로 대기를 투과한 태양광이 대기의 구성 성분이나 지면에 흡수되고, 또 재방출 되는 복잡한 과정을 통하여 온도가 결정되므로 이러한 온도를 복사 온도(radiative temperature)라 한다. 
실제 표면 온도는 행성의 유효온도에 대기의 온실효과 등이 더해져서 결정되어진 온도이다. 

%(https://solarsystem.nasa.gov/system/resources/detail_files/681_ptemp.jpg)