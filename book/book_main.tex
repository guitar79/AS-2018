%%%%%%%%%%%%%%%%%%%%%%%%%%%%%%%%%%%%%%%%%
% The Legrand Orange Book
% LaTeX Template
% Version 2.2 (30/3/17)
%
% This template has been downloaded from:
% http://www.LaTeXTemplates.com
%
% Original author:
% Mathias Legrand (legrand.mathias@gmail.com) with modifications by:
% Vel (vel@latextemplates.com)
%
% License:
% CC BY-NC-SA 3.0 (http://creativecommons.org/licenses/by-nc-sa/3.0/)
%
% Compiling this template:
% This template uses biber for its bibliography and makeindex for its index.
% When you first open the template, compile it from the command line with the 
% commands below to make sure your LaTeX distribution is configured correctly:
%
% 1) pdflatex main
% 2) makeindex main.idx -s StyleInd.ist
% 3) biber main
% 4) pdflatex main x 2
%
% After this, when you wish to update the bibliography/index use the appropriate
% command above and make sure to compile with pdflatex several times 
% afterwards to propagate your changes to the document.
%
% This template also uses a number of packages which may need to be
% updated to the newest versions for the template to compile. It is strongly
% recommended you update your LaTeX distribution if you have any
% compilation errors.
%
% Important note:
% Chapter heading images should have a 2:1 width:height ratio,
% e.g. 920px width and 460px height.
%
%%%%%%%%%%%%%%%%%%%%%%%%%%%%%%%%%%%%%%%%%

%----------------------------------------------------------------------------------------
%	PACKAGES AND OTHER DOCUMENT CONFIGURATIONS
%----------------------------------------------------------------------------------------

%\documentclass[11pt,fleqn]{book_main} % Default font size and left-justified equations
\documentclass[11pt,fleqn]{book} % Default font size and left-justified equations

%----------------------------------------------------------------------------------------

%\input{sub/structure} % Insert the commands.tex file which contains the majority of the structure behind the template
%\include{sub/structure} % Insert the commands.tex file which contains the majority of the
%%%%%%%%%%%%%%%%%%%%%%%%%%%%%%%%%%%%%%%%%
% The Legrand Orange Book
% Structural Definitions File
% Version 2.0 (9/2/15)
%
% Original author:
% Mathias Legrand (legrand.mathias@gmail.com) with modifications by:
% Vel (vel@latextemplates.com)
% 
% This file has been downloaded from:
% http://www.LaTeXTemplates.com
%
% License:
% CC BY-NC-SA 3.0 (http://creativecommons.org/licenses/by-nc-sa/3.0/)
%
%%%%%%%%%%%%%%%%%%%%%%%%%%%%%%%%%%%%%%%%%

%\documentclass{book_main}
%\usepackage{hangul} %<===> 유니코드/UTF-8 
\usepackage[hangul]{kotex} %<===> EUC-KR
\usepackage{ifxetex} % 부득이하게 pdflatex을 사용해도 문제가 없도록 함

\ifxetex
%한글 사용 옵션
\RequirePackage{xetexko}
\setmainfont[Ligatures=TeX]{Batang}
	\setmainhangulfont[BoldFont=*,BoldFeatures=FakeBold,%
ItalicFont=*,ItalicFeatures=FakeSlant]{Batang}
\disablecjksymbolspacing
\nonfrenchspacing
\else
\fi

\usepackage{listings}
\usepackage{color}

\definecolor{dkgreen}{rgb}{0,0.6,0}
\definecolor{gray}{rgb}{0.5,0.5,0.5}
\definecolor{mauve}{rgb}{0.58,0,0.82}

\lstset{frame=tb,
  language=Python,
  aboveskip=3mm,
  belowskip=3mm,
  showstringspaces=false,
  columns=flexible,
  basicstyle={\small\ttfamily},
  numbers=left,                   % where to put the line-numbers
  numberstyle=\tiny\color{gray},  % the style that is used for the line-numbers
  stepnumber=1,                   % the step between two line-numbers. If it's 1, each line
  %numbers=none,
  numberstyle=\tiny\color{gray},
  keywordstyle=\color{blue},
  commentstyle=\color{dkgreen},
  stringstyle=\color{mauve},
  breaklines=true,
  breakatwhitespace=true,
  tabsize=3
}

\citation
\bibdata


\usepackage[margin=10pt,font=small,labelfont=bf]{caption}
\usepackage{subcaption}
%\usepackage{float}
%\floatstyle{plaintop}
%\restylefloat{table}
\usepackage[tableposition=top]{caption}

%%%%%%%%%%%%%%%%%%%%%%%%%%%%%%%%%%%%
%%% caption format %%%%%
%%%%%%%%%%%%%%%%%%%%%%%%%%%%%%%%%%%%
%\renewcommand{\figurename}{Fig.}
\captionsetup[figure]{%
	font=small,labelsep=period,singlelinecheck=off,aboveskip=0pt,% 
	belowskip=5pt, margin=0pt,subrefformat=parens%
}
\captionsetup[table]{%
	font=small,labelsep=period,singlelinecheck=off,aboveskip=5pt,% 
	margin=0pt%
}
\captionsetup[equation]{%
	font=small,labelsep=period,singlelinecheck=off,aboveskip=5pt,% 
	margin=0pt%
}





%----------------------------------------------------------------------------------------
%	VARIOUS REQUIRED PACKAGES AND CONFIGURATIONS
%----------------------------------------------------------------------------------------

\usepackage[top=3cm,bottom=3cm,left=3cm,right=3cm,headsep=10pt,a4paper]{geometry} % Page margins

\usepackage{graphicx} % Required for including pictures
\graphicspath{{Pictures/}} % Specifies the directory where pictures are stored

\usepackage{lipsum} % Inserts dummy text

\usepackage{tikz} % Required for drawing custom shapes

\usepackage[english]{babel} % English language/hyphenation

\usepackage{enumitem} % Customize lists
\setlist{nolistsep} % Reduce spacing between bullet points and numbered lists

\usepackage{booktabs} % Required for nicer horizontal rules in tables

\usepackage{xcolor} % Required for specifying colors by name
\definecolor{ocre}{RGB}{243,102,25} % Define the orange color used for highlighting throughout the book

%----------------------------------------------------------------------------------------
%	FONTS
%----------------------------------------------------------------------------------------

\usepackage{avant} % Use the Avantgarde font for headings
%\usepackage{times} % Use the Times font for headings
\usepackage{mathptmx} % Use the Adobe Times Roman as the default text font together with math symbols from the Sym­bol, Chancery and Com­puter Modern fonts

\usepackage{microtype} % Slightly tweak font spacing for aesthetics
\usepackage[utf8]{inputenc} % Required for including letters with accents
\usepackage[T1]{fontenc} % Use 8-bit encoding that has 256 glyphs

%----------------------------------------------------------------------------------------
%	BIBLIOGRAPHY AND INDEX
%----------------------------------------------------------------------------------------

\usepackage[style=alphabetic,citestyle=numeric,sorting=nyt,sortcites=true,autopunct=true,babel=hyphen,hyperref=true,abbreviate=false,backref=true,backend=biber]{biblatex}
\addbibresource{bibliography.bib} % BibTeX bibliography file
\defbibheading{bibempty}{}

\usepackage{calc} % For simpler calculation - used for spacing the index letter headings correctly
\usepackage{makeidx} % Required to make an index
\makeindex % Tells LaTeX to create the files required for indexing

\bibliography{bibfile} % 참고문헌
% BibTeX 코드 쉽게 얻어오는 방법 %
% Google Scholar 에서 검색한 결과에서 `인용'을 클릭한다.
% BibTeX 코드를 얻고자 한다면, 하단의 `BibTeX' 을 클릭.
% 코드가 나온다. Ctrl+A, Ctrl+C로 복사, bibfile에 붙여넣기.

%----------------------------------------------------------------------------------------
%	MAIN TABLE OF CONTENTS
%----------------------------------------------------------------------------------------

\usepackage{titletoc} % Required for manipulating the table of contents

\contentsmargin{0cm} % Removes the default margin

% Part text styling
\titlecontents{part}[0cm]
{\addvspace{20pt}\centering\large\bfseries}
{}
{}
{}

% Chapter text styling
\titlecontents{chapter}[0em] % Indentation
{\addvspace{12pt}\large\sffamily\bfseries} % Spacing and font options for chapters
{\color{ocre!60}\thecontentslabel\color{ocre}} % Chapter number
{}
{} % Page number
[]

%{\color{ocre!60}\normalsize\;\titlerule*[.5pc]{.}\;\thecontentspage} % Page number



% Section text styling
\titlecontents{section}[1.25cm] % Indentation
{\addvspace{3pt}\sffamily\bfseries} % Spacing and font options for sections
{\contentslabel[\thecontentslabel]{1.25cm}} % Section number
{}
{\ \titlerule*[.5pc]{.}\;\thecontentspage} % Page number
[]

% Subsection text styling
\titlecontents{subsection}[1.25cm] % Indentation
{\addvspace{1pt}\sffamily\small} % Spacing and font options for subsections
{\contentslabel[\thecontentslabel]{1.25cm}} % Subsection number
{}
{\ \titlerule*[.5pc]{.}\;\thecontentspage} % Page number
[]

% List of figures
\titlecontents{figure}[0em]
{\addvspace{5pt}\sffamily}
{\thecontentslabel\hspace*{1em}}
{}
{\ \titlerule*[.5pc]{.}\;\thecontentspage}
[]

% List of tables
\titlecontents{table}[0em]
{\addvspace{5pt}\sffamily}
{\thecontentslabel\hspace*{1em}}
{}
{\ \titlerule*[.5pc]{.}\;\thecontentspage}
[]

%----------------------------------------------------------------------------------------
%	MINI TABLE OF CONTENTS IN PART HEADS
%----------------------------------------------------------------------------------------

% Chapter text styling
\titlecontents{lchapter}[0em] % Indenting
{\addvspace{15pt}\large\sffamily\bfseries} % Spacing and font options for chapters
{\color{ocre}\contentslabel[\Large\thecontentslabel]{1.5cm}\color{ocre}} % Chapter number
{}  
{\color{ocre}\normalsize\sffamily\bfseries\;\titlerule*[.5pc]{.}\;\thecontentspage} % Page number

% Section text styling
\titlecontents{lsection}[0em] % Indenting
{\sffamily\small} % Spacing and font options for sections
{\contentslabel[\thecontentslabel]{1.25cm}} % Section number
{}
{}

% Subsection text styling
\titlecontents{lsubsection}[.5em] % Indentation
{\normalfont\footnotesize\sffamily} % Font settings
{}
{}
{}

%----------------------------------------------------------------------------------------
%	PAGE HEADERS
%----------------------------------------------------------------------------------------

\usepackage{fancyhdr} % Required for header and footer configuration

\pagestyle{fancy}
\renewcommand{\chaptermark}[1]{\markboth{\sffamily\normalsize\bfseries\chaptername\ \thechapter.\ #1}{}} % Chapter text font settings
\renewcommand{\sectionmark}[1]{\markright{\sffamily\normalsize\thesection\hspace{5pt}#1}{}} % Section text font settings
\fancyhf{} \fancyhead[LE,RO]{\sffamily\normalsize\thepage} % Font setting for the page number in the header
\fancyhead[LO]{\rightmark} % Print the nearest section name on the left side of odd pages
\fancyhead[RE]{\leftmark} % Print the current chapter name on the right side of even pages
\renewcommand{\headrulewidth}{0.5pt} % Width of the rule under the header
\addtolength{\headheight}{2.5pt} % Increase the spacing around the header slightly
\renewcommand{\footrulewidth}{0pt} % Removes the rule in the footer
\fancypagestyle{plain}{\fancyhead{}\renewcommand{\headrulewidth}{0pt}} % Style for when a plain pagestyle is specified

% Removes the header from odd empty pages at the end of chapters
\makeatletter
\renewcommand{\cleardoublepage}{
\clearpage\ifodd\c@page\else
\hbox{}
\vspace*{\fill}
\thispagestyle{empty}
\newpage
\fi}

%----------------------------------------------------------------------------------------
%	THEOREM STYLES
%----------------------------------------------------------------------------------------

\usepackage{amsmath,amsfonts,amssymb,amsthm} % For math equations, theorems, symbols, etc
\renewcommand{\vec}[1]{\boldsymbol{#1}}

\newcommand{\intoo}[2]{\mathopen{]}#1\,;#2\mathclose{[}}
\newcommand{\ud}{\mathop{\mathrm{{}d}}\mathopen{}}
\newcommand{\intff}[2]{\mathopen{[}#1\,;#2\mathclose{]}}
\newtheorem{notation}{Notation}[chapter]

% Boxed/framed environments
\newtheoremstyle{ocrenumbox}% % Theorem style name
{0pt}% Space above
{0pt}% Space below
{\normalfont}% % Body font
{}% Indent amount
{\small\bf\sffamily\color{ocre}}% % Theorem head font
{\;}% Punctuation after theorem head
{0.25em}% Space after theorem head
{\small\sffamily\color{ocre}\thmname{#1}\nobreakspace\thmnumber{\@ifnotempty{#1}{}\@upn{#2}}% Theorem text (e.g. Theorem 2.1)
\thmnote{\nobreakspace\the\thm@notefont\sffamily\bfseries\color{black}---\nobreakspace#3.}} % Optional theorem note
\renewcommand{\qedsymbol}{$\blacksquare$}% Optional qed square

\newtheoremstyle{blacknumex}% Theorem style name
{5pt}% Space above
{5pt}% Space below
{\normalfont}% Body font
{} % Indent amount
{\small\bf\sffamily}% Theorem head font
{\;}% Punctuation after theorem head
{0.25em}% Space after theorem head
{\small\sffamily{\tiny\ensuremath{\blacksquare}}\nobreakspace\thmname{#1}\nobreakspace\thmnumber{\@ifnotempty{#1}{}\@upn{#2}}% Theorem text (e.g. Theorem 2.1)
\thmnote{\nobreakspace\the\thm@notefont\sffamily\bfseries---\nobreakspace#3.}}% Optional theorem note

\newtheoremstyle{blacknumbox} % Theorem style name
{0pt}% Space above
{0pt}% Space below
{\normalfont}% Body font
{}% Indent amount
{\small\bf\sffamily}% Theorem head font
{\;}% Punctuation after theorem head
{0.25em}% Space after theorem head
{\small\sffamily\thmname{#1}\nobreakspace\thmnumber{\@ifnotempty{#1}{}\@upn{#2}}% Theorem text (e.g. Theorem 2.1)
\thmnote{\nobreakspace\the\thm@notefont\sffamily\bfseries---\nobreakspace#3.}}% Optional theorem note

% Non-boxed/non-framed environments
\newtheoremstyle{ocrenum}% % Theorem style name
{5pt}% Space above
{5pt}% Space below
{\normalfont}% % Body font
{}% Indent amount
{\small\bf\sffamily\color{ocre}}% % Theorem head font
{\;}% Punctuation after theorem head
{0.25em}% Space after theorem head
{\small\sffamily\color{ocre}\thmname{#1}\nobreakspace\thmnumber{\@ifnotempty{#1}{}\@upn{#2}}% Theorem text (e.g. Theorem 2.1)
\thmnote{\nobreakspace\the\thm@notefont\sffamily\bfseries\color{black}---\nobreakspace#3.}} % Optional theorem note
\renewcommand{\qedsymbol}{$\blacksquare$}% Optional qed square
\makeatother

% Defines the theorem text style for each type of theorem to one of the three styles above
\newcounter{dummy} 
\numberwithin{dummy}{section}
\theoremstyle{ocrenumbox}
\newtheorem{theoremeT}[dummy]{Theorem}
\newtheorem{problem}{Problem}[chapter]
\newtheorem{exerciseT}{Exercise}[chapter]
\theoremstyle{blacknumex}
\newtheorem{exampleT}{Example}[chapter]
\theoremstyle{blacknumbox}
\newtheorem{vocabulary}{Vocabulary}[chapter]
\newtheorem{definitionT}{Definition}[section]
\newtheorem{codeT}{Code}[section]
\newtheorem{corollaryT}[dummy]{Corollary}
\theoremstyle{ocrenum}
\newtheorem{proposition}[dummy]{Proposition}



%----------------------------------------------------------------------------------------
%	DEFINITION OF COLORED BOXES
%----------------------------------------------------------------------------------------

\RequirePackage[framemethod=default]{mdframed} % Required for creating the theorem, definition, exercise and corollary boxes

% Theorem box
\newmdenv[skipabove=7pt,
skipbelow=7pt,
backgroundcolor=black!5,
linecolor=ocre,
innerleftmargin=5pt,
innerrightmargin=5pt,
innertopmargin=5pt,
leftmargin=0cm,
rightmargin=0cm,
innerbottommargin=5pt]{tBox}

% Exercise box	  
\newmdenv[skipabove=7pt,
skipbelow=7pt,
rightline=false,
leftline=true,
topline=false,
bottomline=false,
backgroundcolor=ocre!10,
linecolor=ocre,
innerleftmargin=5pt,
innerrightmargin=5pt,
innertopmargin=5pt,
innerbottommargin=5pt,
leftmargin=0cm,
rightmargin=0cm,
linewidth=4pt]{eBox}	

% Definition box
\newmdenv[skipabove=7pt,
skipbelow=7pt,
rightline=false,
leftline=true,
topline=false,
bottomline=false,
linecolor=ocre,
innerleftmargin=5pt,
innerrightmargin=5pt,
innertopmargin=0pt,
leftmargin=0cm,
rightmargin=0cm,
linewidth=4pt,
innerbottommargin=0pt]{dBox}	

% Corollary box
\newmdenv[skipabove=7pt,
skipbelow=7pt,
rightline=false,
leftline=true,
topline=false,
bottomline=false,
linecolor=gray,
backgroundcolor=black!5,
innerleftmargin=5pt,
innerrightmargin=5pt,
innertopmargin=5pt,
leftmargin=0cm,
rightmargin=0cm,
linewidth=4pt,
innerbottommargin=5pt]{cBox}

% Creates an environment for each type of theorem and assigns it a theorem text style from the "Theorem Styles" section above and a colored box from above
\newenvironment{theorem}{\begin{tBox}\begin{theoremeT}}{\end{theoremeT}\end{tBox}}
\newenvironment{exercise}{\begin{eBox}\begin{exerciseT}}{\hfill{\color{ocre}\tiny\ensuremath{\blacksquare}}\end{exerciseT}\end{eBox}}				  
\newenvironment{definition}{\begin{dBox}\begin{definitionT}}{\end{definitionT}\end{dBox}}	
\newenvironment{example}{\begin{exampleT}}{\hfill{\tiny\ensuremath{\blacksquare}}\end{exampleT}}		
\newenvironment{code}{\begin{eBox}\begin{codeT}}{\hfill{\color{ocre}\tiny\ensuremath{\blacksquare}}\end{codeT}\end{eBox}}				  

%----------------------------------------------------------------------------------------
%	REMARK ENVIRONMENT
%----------------------------------------------------------------------------------------

\newenvironment{remark}{\par\vspace{10pt}\small % Vertical white space above the remark and smaller font size
\begin{list}{}{
\leftmargin=35pt % Indentation on the left
\rightmargin=25pt}\item\ignorespaces % Indentation on the right
\makebox[-2.5pt]{\begin{tikzpicture}[overlay]
\node[draw=ocre!60,line width=1pt,circle,fill=ocre!25,font=\sffamily\bfseries,inner sep=2pt,outer sep=0pt] at (-15pt,0pt){\textcolor{ocre}{R}};\end{tikzpicture}} % Orange R in a circle
\advance\baselineskip -1pt}{\end{list}\vskip5pt} % Tighter line spacing and white space after remark

%----------------------------------------------------------------------------------------
%	SECTION NUMBERING IN THE MARGIN
%----------------------------------------------------------------------------------------

\makeatletter
\renewcommand{\@seccntformat}[1]{\llap{\textcolor{ocre}{\csname the#1\endcsname}\hspace{1em}}}                    
\renewcommand{\section}{\@startsection {section}{1}{\z@}
{-4ex \@plus -1ex \@minus -.4ex}
{1ex \@plus.2ex }
{\normalfont\large\sffamily\bfseries}}
\renewcommand{\subsection}{\@startsection {subsection}{2}{\z@}
{-3ex \@plus -0.1ex \@minus -.4ex}
{0.5ex \@plus.2ex }
{\normalfont\sffamily\bfseries}}
\renewcommand{\subsubsection}{\@startsection {subsubsection}{3}{\z@}
{-2ex \@plus -0.1ex \@minus -.2ex}
{.2ex \@plus.2ex }
{\normalfont\small\sffamily\bfseries}}                        
\renewcommand\paragraph{\@startsection{paragraph}{4}{\z@}
{-2ex \@plus-.2ex \@minus .2ex}
{.1ex}
{\normalfont\small\sffamily\bfseries}}


%%%%%%%%%%%%%%%%%%%%%%%%%%%%%%%%%%%%%%%
%%% Chapter/section title format %%%%%%
%%%%%%%%%%%%%%%%%%%%%%%%%%%%%%%%%%%%%%%
% ``각 절 제목의 글자는 section (21pt, bold face), subsection (16pt, bold face), subsubsection (14pt, normal style), paragraph (12pt, Italic)로 설정한다.''
% ``각 절의 첫 문단은 들여쓰기를 하지 않지만, 두 번째 문단 이후부터는 들여쓰기 (16pt)를 한다. 각 절의 제목에는 글씨 크기에 비례하는 상하 여백을 준다.''
% 졸업논문 hwp 양식에서 따옴
%\titleformat{\section}[hang]
%{\normalfont\fontsize{21}{21}\selectfont\bfseries}{\arabic{section}.}{1em}{}
%%\titleformat{\subsection}[hang]
%%{\normalfont\fontsize{16}{16}\selectfont\bfseries}{\arabic{section}.\arabic{subsection}}{1em}{}
%%\titleformat{\subsubsection}[hang]
%%{\normalfont\fontsize{14}{14}\selectfont}{\arabic{section}.\arabic{subsection}.\arabic{subsubsection}}{1em}{}
%%\titleformat{\paragraph}[hang]
%%{\normalfont\fontsize{12}{12}\selectfont\it}{}{1em}{}


%----------------------------------------------------------------------------------------
%	PART HEADINGS
%----------------------------------------------------------------------------------------

% numbered part in the table of contents
\newcommand{\@mypartnumtocformat}[2]{%
\setlength\fboxsep{0pt}%
\noindent\colorbox{ocre!20}{\strut\parbox[c][.7cm]{\ecart}{\color{ocre!70}\Large\sffamily\bfseries\centering#1}}\hskip\esp\colorbox{ocre!40}{\strut\parbox[c][.7cm]{\linewidth-\ecart-\esp}{\Large\sffamily\centering#2}}}%
%%%%%%%%%%%%%%%%%%%%%%%%%%%%%%%%%%
% unnumbered part in the table of contents
\newcommand{\@myparttocformat}[1]{%
\setlength\fboxsep{0pt}%
\noindent\colorbox{ocre!40}{\strut\parbox[c][10.7cm]{\linewidth}{\Large\sffamily\centering#1}}}%
%%%%%%%%%%%%%%%%%%%%%%%%%%%%%%%%%%
\newlength\esp
\setlength\esp{4pt}
\newlength\ecart
\setlength\ecart{1.2cm-\esp}
\newcommand{\thepartimage}{}%
\newcommand{\partimage}[1]{\renewcommand{\thepartimage}{#1}}%
\def\@part[#1]#2{%
\ifnum \c@secnumdepth >-2\relax%
\refstepcounter{part}%
\addcontentsline{toc}{part}{\texorpdfstring{\protect\@mypartnumtocformat{\thepart}{#1}}{\partname~\thepart\ ---\ #1}}
\else%
\addcontentsline{toc}{part}{\texorpdfstring{\protect\@myparttocformat{#1}}{#1}}%
\fi%
\startcontents%
\markboth{}{}%
{\thispagestyle{empty}%
\begin{tikzpicture}[remember picture,overlay]%
\node at (current page.north west){\begin{tikzpicture}[remember picture,overlay]%	
\fill[ocre!20](0cm,0cm) rectangle (\paperwidth,-\paperheight);
\node[anchor=north] at (4cm,-3.25cm){\color{ocre!40}\fontsize{220}{100}\sffamily\bfseries\thepart}; 
\node[anchor=south east] at (\paperwidth-1cm,-\paperheight+1cm){\parbox[t][][t]{8.5cm}{
\printcontents{l}{0}{\setcounter{tocdepth}{1}}%
}};
\node[anchor=north east] at (\paperwidth-1.5cm,-3.25cm){\parbox[t][][t]{15cm}{\strut\raggedleft\color{white}\fontsize{30}{30}\sffamily\bfseries#2}};
\end{tikzpicture}};
\end{tikzpicture}}%
\@endpart}
\def\@spart#1{%
\startcontents%
\phantomsection
{\thispagestyle{empty}%
\begin{tikzpicture}[remember picture,overlay]%
\node at (current page.north west){\begin{tikzpicture}[remember picture,overlay]%	
\fill[ocre!20](0cm,0cm) rectangle (\paperwidth,-\paperheight);
\node[anchor=north east] at (\paperwidth-1.5cm,-3.25cm){\parbox[t][][t]{15cm}{\strut\raggedleft\color{white}\fontsize{30}{30}\sffamily\bfseries#1}};
\end{tikzpicture}};
\end{tikzpicture}}
\addcontentsline{toc}{part}{\texorpdfstring{%
\setlength\fboxsep{0pt}%
\noindent\protect\colorbox{ocre!40}{\strut\protect\parbox[c][.7cm]{\linewidth}{\Large\sffamily\protect\centering #1\quad\mbox{}}}}{#1}}%
\@endpart}
\def\@endpart{\vfil\newpage
\if@twoside
\if@openright
\null
\thispagestyle{empty}%
\newpage
\fi
\fi
\if@tempswa
\twocolumn
\fi}

%----------------------------------------------------------------------------------------
%	CHAPTER HEADINGS
%----------------------------------------------------------------------------------------

% A switch to conditionally include a picture, implemented by  Christian Hupfer
\newif\ifusechapterimage
\usechapterimagetrue
\newcommand{\thechapterimage}{}%
\newcommand{\chapterimage}[1]{\ifusechapterimage\renewcommand{\thechapterimage}{#1}\fi}%
\newcommand{\autodot}{.}
\def\@makechapterhead#1{%
{\parindent \z@ \raggedright \normalfont
\ifnum \c@secnumdepth >\m@ne
\if@mainmatter
\begin{tikzpicture}[remember picture,overlay]
\node at (current page.north west)
{\begin{tikzpicture}[remember picture,overlay]
\node[anchor=north west,inner sep=0pt] at (0,0) {\ifusechapterimage\includegraphics[width=\paperwidth]{\thechapterimage}\fi};
\draw[anchor=west] (\Gm@lmargin,-9cm) node [line width=2pt,rounded corners=15pt,draw=ocre,fill=white,fill opacity=0.5,inner sep=15pt]{\strut\makebox[22cm]{}};
\draw[anchor=west] (\Gm@lmargin+.3cm,-9cm) node {\huge\sffamily\bfseries\color{black}\thechapter\autodot~#1\strut};
\end{tikzpicture}};
\end{tikzpicture}
\else
\begin{tikzpicture}[remember picture,overlay]
\node at (current page.north west)
{\begin{tikzpicture}[remember picture,overlay]
\node[anchor=north west,inner sep=0pt] at (0,0) {\ifusechapterimage\includegraphics[width=\paperwidth]{\thechapterimage}\fi};
\draw[anchor=west] (\Gm@lmargin,-9cm) node [line width=2pt,rounded corners=15pt,draw=ocre,fill=white,fill opacity=0.5,inner sep=15pt]{\strut\makebox[22cm]{}};
\draw[anchor=west] (\Gm@lmargin+.3cm,-9cm) node {\huge\sffamily\bfseries\color{black}#1\strut};
\end{tikzpicture}};
\end{tikzpicture}
\fi\fi\par\vspace*{270\p@}}}

%-------------------------------------------

\def\@makeschapterhead#1{%
\begin{tikzpicture}[remember picture,overlay]
\node at (current page.north west)
{\begin{tikzpicture}[remember picture,overlay]
\node[anchor=north west,inner sep=0pt] at (0,0) {\ifusechapterimage\includegraphics[width=\paperwidth]{\thechapterimage}\fi};
\draw[anchor=west] (\Gm@lmargin,-9cm) node [line width=2pt,rounded corners=15pt,draw=ocre,fill=white,fill opacity=0.5,inner sep=15pt]{\strut\makebox[22cm]{}};
\draw[anchor=west] (\Gm@lmargin+.3cm,-9cm) node {\huge\sffamily\bfseries\color{black}#1\strut};
\end{tikzpicture}};
\end{tikzpicture}
\par\vspace*{270\p@}}
\makeatother

%----------------------------------------------------------------------------------------
%	HYPERLINKS IN THE DOCUMENTS
%----------------------------------------------------------------------------------------

\usepackage{hyperref}
\hypersetup{hidelinks,backref=true,pagebackref=true,hyperindex=true,colorlinks=false,breaklinks=true,urlcolor= ocre,bookmarks=true,bookmarksopen=false,pdftitle={Title},pdfauthor={Author}}
\usepackage{bookmark}
\bookmarksetup{
open,
numbered,
addtohook={%
\ifnum\bookmarkget{level}=0 % chapter
\bookmarksetup{bold}%
\fi
\ifnum\bookmarkget{level}=-1 % part
\bookmarksetup{color=ocre,bold}%
\fi
}
}
 % Insert the commands.tex file which contains the majority of the structure behind the template
\begin{document}


%----------------------------------------------------------------------------------------
%	TITLE PAGE
%----------------------------------------------------------------------------------------

\begingroup
\thispagestyle{empty}
\begin{tikzpicture}[remember picture,overlay]
\node[inner sep=0pt] (background) at (current page.center) {\includegraphics[width=\paperwidth]{background}};
\draw (current page.center) node [fill=ocre!30!white,fill opacity=0.6,text opacity=1,inner sep=1cm]{\Huge\centering\bfseries\sffamily\parbox[c][][t]{\paperwidth}{\centering 대기 과학\\[15pt] % Book title
		{\Large Atmospheric science}\\[20pt] % Subtitle
		{\huge 박 기 현 }}}; % Author name
\end{tikzpicture}
\vfill
\endgroup

%----------------------------------------------------------------------------------------
%	COPYRIGHT PAGE
%----------------------------------------------------------------------------------------

\newpage
~\vfill
\thispagestyle{empty}

\noindent Copyright \copyright\ 2017 Park, Kie-hyun\\ % Copyright notice

\noindent \textsc{Published by 경기과학고등학교}\\ % Publisher

\noindent \textsc{www.gs.hs.kr}\\ % URL

\noindent Licensed under the Creative Commons Attribution-NonCommercial 3.0 Unported License (the ``License''). You may not use this file except in compliance with the License. You may obtain a copy of the License at \url{http://creativecommons.org/licenses/by-nc/3.0}. Unless required by applicable law or agreed to in writing, software distributed under the License is distributed on an \textsc{``as is'' basis, without warranties or conditions of any kind}, either express or implied. See the License for the specific language governing permissions and limitations under the License.\\ % License information

\noindent \textit{First printing, August 2017} % Printing/edition date

%----------------------------------------------------------------------------------------
%	TABLE OF CONTENTS
%----------------------------------------------------------------------------------------

%\usechapterimagefalse % If you don't want to include a chapter image, use this to toggle images off - it can be enabled later with \usechapterimagetrue

\chapterimage{chapter_head_1.pdf} % Table of contents heading image

\pagestyle{empty} % No headers

\tableofcontents % Print the table of contents itself

\cleardoublepage % Forces the first chapter to start on an odd page so it's on the right

\pagestyle{fancy} % Print headers again
 % Title
%%%%%%%%%%%%%%%%%%%%%%%%%%%%%%%%%%%%%%%%%%%%%%%%%%%%%%%%%%%
%%%% Main Document %%%%%%%%%%%%%%%%%%%%%%%%%%%%%%%%%%%%%%%%
%%%%%%%%%%%%%%%%%%%%%%%%%%%%%%%%%%%%%%%%%%%%%%%%%%%%%%%%%%%
% \include{sub/methodology} 와 같이 작성
%%%% 주의
%%%% 파일이 나뉠 때마다 자동으로 페이지넘김(\clearpage)가 됩니다. 
%%%% 따라서 subsection을 나누는 용도로는 사용하지 마십시오.
%%%% \include{sub/experiment} 와 같이...
%----------------------------------------------------------------------------------------
%	PART
%----------------------------------------------------------------------------------------

\part{대기 역학}

%----------------------------------------------------------------------------------------
%	CHAPTER
%----------------------------------------------------------------------------------------

\chapterimage{chapter_head_2.pdf} % Chapter heading image

\chapter{대기 역학}\index{대기 역학}

%%%%%%%%%%%%%%%%%%%%%%%%%%%%%%%%%%%%%%%%%%%%%%%%%%%%%%%%%%%


\section{좌표계}

\subsection{좌표계1}

관성계 : 절대 좌표계 $ (x,~y,~z)$ \\
비관성계 : 회전좌표계 $ (x^{\prime},~y^{\prime},~z^{\prime})$ 
극좌표 $ (r,~\theta,~z)$\\
\\
$ (x,~y,~z) 	\rightarrow (r,~\theta,~z)$ 에서 \\
수평 방향은 정역학 평형 상태에 있으므로, \\	

$ (x, y) 	\rightarrow (r, \theta)$\\

$ \mathbf{F} = F_{x} \mathbf{\hat{i}}  + F_{y} \mathbf{\hat{j}} $ 에서 
$ F_{x} = m \dfrac{d^{2}x}{dt^{2}}$, 
$ F_{y} = m \dfrac{d^{2}y}{dt^{2}}$ 라고 할 수 있다.  \\

$ (x, y) = (r \cos \theta, r \sin \theta)$ 에서 \\
$ F_{r} = F_{x} \cos \theta + F_{y} \sin \theta $, 
$ F_{\theta} = F_{y} \cos \theta - F_{x} \cos \theta $ 로 나타낼 수 있다. \\

$ x = r \cos \theta $를 미분하면, \\
$\dfrac{dx}{dt} = \cos \theta \dfrac{dr}{dt} - r \sin \theta \dfrac{d\theta}{dt}$ 이고, \\
이를 다시 미분하면, 
$\dfrac{d^{2}x}{dt^{2}} = \cos \theta \dfrac{d^{2}r}{dt^{2}} - \sin \theta \dfrac{dr}{dt} - \sin \theta \dfrac{dr}{dt} \dfrac{d\theta}{dt} -r \cos \theta \dfrac{d^{2}\theta}{dt^{2}}$\\
같은 방법으로 $ y = r \sin \theta $ 를 미분하면 \\
$\dfrac{dy}{dt} = \sin \theta \dfrac{dr}{dt} + r \cos \theta \dfrac{d\theta}{dt}$ 이고,\\
이를 다시 미분하면
$\dfrac{d^{2}t}{dt^{2}} = \sin \theta \dfrac{d^{2}r}{dt^{2}} + \cos \theta \dfrac{dr}{dt} + \cos \theta \dfrac{dr}{dt} \dfrac{d\theta}{dt} -r \sin \theta \dfrac{d^{2}\theta}{dt^{2}}$이다. \\



$ F_{r} = F_{x} \cos \theta + F_{y} \sin \theta 
= m \left ( \dfrac{d^{2}x}{dt^{2}} \cos \theta + \dfrac{d^{2}y}{dt^{2}} \sin \theta \right) $\\

$ F_{\theta} = F_{y} \cos \theta - F_{x} \cos \theta 
= m \left ( \dfrac{d^{2}y}{dt^{2}} \cos \theta - \dfrac{d^{2}x}{dt^{2}} \cos \theta \right) $\\


정리하면, \\

$ F_{r} = m \left[ \dfrac{d^{2}r}{dt^{2}} - r \left( {\dfrac{d \theta}{dt}} \right)^{2} \right] $에서, 
$ -r \left( {\dfrac{d \theta}{dt}} \right)^{2} \rightarrow $ Centrifugal force \\

$ F_{\theta} = m \left[ r \dfrac{d^{2}\theta}{dt^{2}} + 2 \dfrac{dr}{dt} \dfrac{d\theta}{dt}  \right] $에서, 
$ 2 \dfrac{dr}{dt} \dfrac{d\theta}{dt} \rightarrow $ Coriolis force \\


\subsection{좌표계2}

$ (x,~y) 	\rightarrow (x^{\prime},~y^{\prime})$\\

$ \mathbf {F} = F_{x} \mathbf{\hat{i}} + F_{y} \mathbf{\hat{j}} $에서 
$ F_{x} = m \dfrac{d^{2}x}{dt^{2}}$, 
$ F_{y} = m \dfrac{d^{2}y}{dt^{2}}$ 라고 할 수 있다.\\

$ x^{\prime} = x \cos \omega t + y \sin \omega t$, 
$ y^{\prime} = -x \sin \omega t + y \cos \omega t$\\

$ \mathbf {F} = F_{x^{\prime}} \mathbf {\hat{i}}  + F_{y^{\prime}} \mathbf {\hat{j}} $\\

$ F_{x^{\prime}} = F_{x} \cos \omega t + F_{y} \sin \omega t
= m \left ( \dfrac{d^{2}x}{dt^{2}} \cos \omega t + \dfrac{d^{2}y}{dt^{2}} \sin \omega t \right) $\\

$ F_{y^{\prime}} = -F_{x} \sin \omega t + F_{y} \cos \omega t
= m \left ( - \dfrac{d^{2}x}{dt^{2}} \sin \Omega t + \dfrac{d^{2}y}{dt^{2}} \cos \omega t \right) $\\

정리하면,


$ F_{x^{\prime}} = m \left( \dfrac{d^{2}x}{dt^{2}}F_{x} - 2 \omega  \dfrac{dy}{dt} - 2 \omega^{2} x^{\prime}  \right) $\\

$ F_{y^{\prime}} = m \left( \dfrac{d^{2}y}{dt^{2}}F_{x} + 2 \omega  \dfrac{dx}{dt} - 2 \omega^{2} y^{\prime}  \right) $\\



수정 요함...

$ x^{\prime} = x \cos \Omega t + y \sin \Omega t$ 을 미분하면,\\

$\dfrac{dx^{\prime}}{dt} = \dfrac{dx}{dt} \cos \Omega t - x \sin \Omega t +  \dfrac{dy}{dt} \sin \Omega t + y \cos \Omega t $ \\

$\dfrac{d^{2}x}{dt^{2}} = \cos \theta \dfrac{d^{2}r}{dt^{2}} - \sin \theta \dfrac{dr}{dt} - \sin \theta \dfrac{dr}{dt} \dfrac{d\theta}{dt} -r \cos \theta \dfrac{d^{2}\theta}{dt^{2}}$\\
\\
$ y = r \sin \theta $ \\

$\dfrac{dy}{dt} = \sin \theta \dfrac{dr}{dt} + r \cos \theta \dfrac{d\theta}{dt}$ \\

$\dfrac{d^{2}t}{dt^{2}} = \sin \theta \dfrac{d^{2}r}{dt^{2}} + \cos \theta \dfrac{dr}{dt} + \cos \theta \dfrac{dr}{dt} \dfrac{d\theta}{dt} -r \sin \theta \dfrac{d^{2}\theta}{dt^{2}}$\\
\\



여기까지 수정 요함...


\subsection{각운동량 보존}

$\dfrac{d \theta}{dt} = \Omega$ \\

$ r\dfrac{d \theta}{dt} = r \Omega = u_{\theta} $\\

$\dfrac{dr}{dt} = v_{r}$\\

$\dfrac{d}{dt} r^{2}\Omega = 2 r \dfrac{dr}{dt} \Omega + r^{2} \dfrac{d\Omega}{dt} 
= r \left( r \dfrac{d\Omega}{dt} +2 \dfrac{dr}{dt} \Omega \right) $\\

$ r F_{\theta} = m \dfrac{d}{dt} \left(r^{2} \Omega \right) $\\

$ r^{2} \Omega = const$\\

$ r \left( r \Omega \right) = r u_{\theta} = const $ \\


$ R_{1} V_{1} = R_{2} V_{2}$\\
\\


\subsection{Pressure gradient force}

$ dV = dx \cdot dy \cdot dz $\\

$x$ 방향 \\

$ F_{x} = P \cdot \Delta y \cdot \Delta z - \left( P + \Delta P \right) \Delta y \cdot \Delta z$\\

$ F_{x} = - \Delta P \cdot \Delta y \cdot \Delta z $\\

$ \dfrac { \Delta y}{\Delta x } = \dfrac {f\left(x + \Delta x \right) - f\left(x \right)}{ \Delta x}$\\

$f^{\prime} \left(x \right) = lim \dfrac { \Delta y}{\Delta x } 
= \dfrac {f\left(x + \Delta x \right) - f\left(x \right)}{ \Delta x}$\\

$z = f \left( x, y \right) $ \\
$y = b \rightarrow $ 고정 \\


$\dfrac{\partial z}{\partial x} = \displaystyle \lim_{\Delta x \rightarrow 0} \dfrac { \Delta z}{\Delta x } 
= \dfrac {f\left(x + \Delta x, b \right) - f\left(x, b \right)}{ \Delta x}$\\

$\dfrac{\partial z}{\partial y} = \displaystyle \lim_{\Delta y \rightarrow 0} \dfrac { \Delta z}{\Delta y } 
= \dfrac {f\left(y + \Delta y, b \right) - f\left(y, b \right)}{ \Delta y}$\\

$ \Delta z = \dfrac{\partial z}{\partial x} \Delta x + \dfrac{\partial z}{\partial y} \Delta y $\\

$ dz = \dfrac{\partial z}{\partial x} dx + \dfrac{\partial z}{\partial y} dy $\\


$ \Delta T = \dfrac{\partial T}{\partial t} \Delta t 
+ \dfrac{\partial T}{\partial x} \Delta x 
+ \dfrac{\partial T}{\partial y} \Delta y 
+ \dfrac{\partial T}{\partial z} \Delta z $\\

$ F_{x} = - \Delta P \cdot \Delta y \cdot \Delta z 
= \dfrac{\partial P}{\partial x} \cdot \Delta x \cdot \Delta y \cdot \Delta z $\\

$ F_{y} = - \Delta P \cdot \Delta z \cdot \Delta x 
= \dfrac{\partial P}{\partial y} \cdot \Delta x \cdot \Delta y \cdot \Delta z $\\

$ F_{z} = - \Delta P \cdot \Delta x \cdot \Delta y 
= \dfrac{\partial P}{\partial z} \cdot \Delta x \cdot \Delta y \cdot \Delta z $\\

$ \rho = \dfrac {m}{\Delta x \cdot \Delta y \cdot \Delta z} $ \\

$ \dfrac {F_{x}}{m} = - \dfrac{1}{\rho} \dfrac{\partial P}{\partial x} $ \\

$ \dfrac {F}{m} = - \dfrac{1}{\rho} \left( \dfrac{\partial P}{\partial x} i + \dfrac{\partial P}{\partial y} j + \dfrac{\partial P}{\partial z} k \right) 
= - \dfrac{1}{\rho} \nabla P$ \\


\subsection{Gravity}\

\subsection{회전계에서의 운동 방정식}


$ \Delta T = \dfrac{\partial T}{\partial t} \Delta t 
+ \dfrac{\partial T}{\partial x} \Delta x 
+ \dfrac{\partial T}{\partial y} \Delta y 
+ \dfrac{\partial T}{\partial z} \Delta z $\\

이 식을 $ \Delta T $로 나누고 0으로 극한을 취하면,\\

$ \displaystyle \lim_{\Delta t \rightarrow 0} \dfrac{\Delta T}{\Delta t} 
= \dfrac{DT}{Dt} = \dfrac{\partial T}{\partial t} 
+ \dfrac{\partial T}{\partial x} \dfrac{Dx}{Dt}
+ \dfrac{\partial T}{\partial y} \dfrac{Dy}{Dt}
+ \dfrac{\partial T}{\partial z} \dfrac{Dz}{Dt} $\\

$\dfrac{Dx}{Dt} \equiv u$, 
$\dfrac{Dy}{Dt} \equiv v$, 
$\dfrac{Dz}{Dt} \equiv w$, 
 라고 정의하면\\

$ \dfrac{DT}{Dt} = \dfrac{\partial T}{\partial t} 
+ \left( u \dfrac{\partial T}{\partial x}
+ v \dfrac{\partial T}{\partial y}
+ w \dfrac{\partial T}{\partial z} \right)
= \dfrac{\partial T}{\partial t} + U \cdot \nabla T $\\

여기에서 $ U = iu + jv + kw $ 3차원 속도 벡터 이다.

회전계에서의 운동방정식을 유도하면,

$ \dfrac{DU}{Dt} = -2 \Omega \times U - \dfrac{1}{\rho} \nabla p + g + F_{r} $\\

와 같이 나타낼 수 있다.



\subsection{직각 카테시안 좌표계에서의 운동 방정식}

$ \dfrac{DU}{Dt} = -2 \Omega \times U - \dfrac{1}{\rho} \nabla p + g + F_{r} $ 에서 \\

먼저 전향력 성분을 나누어 보면, 

$ \Omega_{x} = 0$,
$ \Omega_{y} = \Omega \cos \phi$, 
$ \Omega_{z} = \Omega \sin \phi$ 이다.\\

$ -2 \Omega \times U  
= -2 \left| \begin{array}{ccc}
i & j & k \\
0 & \Omega \cos \phi & \Omega \sin \phi \\
u & v & w \end{array} \right| \\

= -2 \left( 2 \Omega w \cos \phi -2 \Omega v \sin \phi \right) \mathbf{i}
- 2 \Omega u \sin \phi \mathbf{j}
+ 2 \Omega u \cos \phi \mathbf{k}$\\
로 나타낼 수 있다. \\

그리고 기압경도력을 나누어 보면, \\

$ \nabla p = \mathbf{i} \dfrac{\partial P}{\partial x} 
+ \mathbf{j} \dfrac{\partial P}{\partial y}
+ \mathbf{k} \dfrac{\partial P}{\partial z}$\\

중력은 
$ \mathbf{g} = -g \mathbf{k} $\\

마찰은 
$ F_{r} = \mathbf{i} F_{x}
+ \mathbf{j} F_{y}
+ \mathbf{k} F_{z}$\\

각 성분별로 운동방정식을 나타내면 \\

$ \dfrac{Du}{Dt}
= - \dfrac{1}{\rho} \dfrac{\partial P}{\partial x} 
+ 2 \Omega v \sin \phi - 2 \Omega w \cos \phi 
+ F_{x} $\\

$ \dfrac{Dv}{Dt}
= - \dfrac{1}{\rho} \dfrac{\partial P}{\partial x} 
- 2 \Omega u \sin \phi
+ F_{y} $\\

$ \dfrac{Dw}{Dt}
= - \dfrac{1}{\rho} \dfrac{\partial P}{\partial x} 
+ 2 \Omega u \cos \phi 
+ F_{z}$\\

$x$ 성분에서 연직 전향력은 수평 전향력에 비해 매우 작은 값이므로, $- 2 \Omega w \cos \phi $ 항을 무시할 수 있다. \\
$z$ 성분의 전향력 $ 2 \Omega u \cos \phi $ 은 중력 $g$에 비해 매우 작으므로 무시할 수 있다. \\
더구나 $ \dfrac{Dw}{Dt}$의 크기는 더 작기 때문에 $ 2 \Omega \sin \phi $를 $f$로 두면 다음과 같이 간단히 할 수 있다.

$ \dfrac{Du}{Dt}
= - \dfrac{1}{\rho} \dfrac{\partial P}{\partial x} 
+ 2 \Omega v \sin \phi 
= - \dfrac{1}{\rho} \dfrac{\partial P}{\partial x} 
+ f v $\\

$ \dfrac{Dv}{Dt}
= - \dfrac{1}{\rho} \dfrac{\partial P}{\partial x} 
- 2 \Omega u \sin \phi
= - \dfrac{1}{\rho} \dfrac{\partial P}{\partial x}
- f u $\\

$ 0
= - \dfrac{1}{\rho} \dfrac{\partial P}{\partial x} 
- g $\\ 


\subsection{자연 좌표계}

자연 좌표계 $ (s,~n,~z)$ \\

$ t $ : 유체가 움직이는 방향에 평행인 방향\\
$ n $ : $ t $에 대하여 수직인 벡터이고 유체가 움직이는 방향의 왼쪽으로 향하는 방향이 + 방향임\\
$ k $ : 연직 방향\\

$\mathbf{V} = V \mathbf{t}$, $\mathbf{V} = \dfrac{Ds}{Dt}$ \\
가속도  $ \dfrac{D\mathbf{V}}{Dt} = \mathbf{t} \dfrac{DV}{Dt} + V \dfrac{D \mathbf{t}}{Dt}$$



\subsection{연습 문제}

1. (1) \\
각운동량 보존법칙은 
$ R_{1}~ V_{1} = R_{2}~ V_{2}$ \\
$ V_{1} = R_{1}~ \Omega_{1}$,
$ V_{2} = R_{2}~ \Omega_{2}$ 이므로
$ {R_{1}}^{2}~ \Omega_{1} = {R_{2}}^{2}~ \Omega_{2}$ \\
$ {1}^2 \times 2 = {0.5}^2 \times \Omega_{2}$ 에서 
$ \Omega_{2} = 8 \left( \rm rad ~ s^{-1} \right) $\\
회전 선속도는 
$ V_{2} = 0.5 \times 8 = 4 \left( \rm m ~ s^{-1} \right) $\\
각운동량은
$ {L}_{2} = R_{2} \times m V_{2} 
= 0.5 \cdot 1 \cdot 4 
= 2 \left( \rm kg ~ m ~ s^{-1} \right)$\\

1. (2) \\
구심가속력은
$ - m ~ {R}_{2} ~ {{\Omega}_{2}^{2}
= - 1 \cdot 0.5 \cdot 8
= - 4 \left( \rm kg ~ m ~ s^{-1} \right)$\\

2.  \\

북위 $37.5^{\circ}$의 자전 선속도는
$R_{E} \cdot \cos \phi \Omega
= 6380000 \cdot 0.7934 \cdot 7.272 \times 10^{-5}
= 368.11 \left( \rm m ~ s^{-1} \right)$\\
절대 좌표계에서 이 바람을 관측한다면 자전 선속도와 바람의 방향이 같으므로\\
$368.11 + 5 = 373.11 \left( \rm m ~ s^{-1} \right)$ 이다.\\

3.  \\

전향력은 \\
$ 2 m v \Omega \sin \phi
= \  2 \cdot 65 \cdot 1000000 \div 3600 \cdot 7.272 \cdot 0.5
= 131300 \left( \rm kg~m ~ s^{-2} = N \right) $이다.\\

4.  \\

지균풍의 풍속은 \\
$ f v_{g} = - {\rho} \dfrac{\partial P}{\partial n}\\

v_{g} = - \dfrac{1}{f} \cdot \dfrac{1}{\rho} \cdot \dfrac{\partial P}{\partial n}
= - \dfrac{1}{2 \Omega \sin \phi } \cdot \dfrac{1}{\rho} \dfrac{\partial P}{\partial n} \\
= - \dfrac{1}{\rho} \cdot \dfrac{1}{2 \cdot 7.272 \left( \rm s^{-1} \right) \cdot 0.5 } \cdot \dfrac{200 \left( \rm kg~m~s^{-2} m^{-2} \right)} {100000 \left( \rm m \right)} \\
= - \dfrac{1}{\rho} \cdot 27.50 \left( \rm kg~m^{-2}~ s^{-1} \right) $이다. 공기의 밀도를 $ 1\rm kg~m^{-3}$이라고 가정하면, \\
$v_{g} = - 27.50 \left( \rm m~ s^{-1} \right) $이다.\\

















\part{부록}


%----------------------------------------------------------------------------------------
%	INDEX
%----------------------------------------------------------------------------------------

\cleardoublepage
\phantomsection
\setlength{\columnsep}{0.75cm}
\addcontentsline{toc}{chapter}{\textcolor{ocre}{Index}}
\printindex

%----------------------------------------------------------------------------------------

%----------------------------------------------
%     List of Figures/Tables (자동 작성됨)
%----------------------------------------------
\cleardoublepage
\clearpage
\listoftables
% 표 목록과 캡션을 출력한다. 만약 논문에 표가 없다면 이 위 줄의 맨 앞에 
% `%' 기호를 넣어서 주석 처리한다.

\cleardoublepage
\clearpage
\listoffigures
% 그림 목록과 캡션을 출력한다. 만약 논문에 그림이 없다면 이 위 줄의 맨 앞에 
% `%' 기호를 넣어서 주석 처리한다.

\cleardoublepage
\clearpage
%\listofequations
% 그림 목록과 캡션을 출력한다. 만약 논문에 그림이 없다면 이 위 줄의 맨 앞에 
% `%' 기호를 넣어서 주석 처리한다.


%----------------------------------------------------------------------------------------
%	BIBLIOGRAPHY
%----------------------------------------------------------------------------------------

\chapter*{Bibliography}
\addcontentsline{toc}{chapter}{\textcolor{ocre}{Bibliography}}
\section*{Books}
\addcontentsline{toc}{section}{Books}
\printbibliography[heading=bibempty,type=book]
\section*{Articles}
\addcontentsline{toc}{section}{Articles}
\printbibliography[heading=bibempty,type=article]


%-----------------------------------------------------
%   감사의 글
%-----------------------------------------------------
\begin{acknowledgements}
\addcontentsline{toc}{section}{감사의 글}  %%% TOC에 표시
정말 감사합니다.
\end{acknowledgements}

%-----------------------------------------------------
%   약력
%-----------------------------------------------------
\begin{career}
\addcontentsline{toc}{section}{경력}  %%% TOC에 표시
\begin{itemize}
\item{.}
\end{itemize}
\end{career}


\end{document}
