\section{대기 정역학 }

# 연습 문제 1

-1. 화씨 온도 눈금은 얼음이 녹는점을 32F 로 물의 끓는점을 212F로 지정하였다. 섭씨와 화씨의 눈금 사이의 관계식을 유도하고, 섭씨 온도가 -40, -30, -20,  -10, 0, 10, 20, 30, 40일 때의 화씨 온도를 구하여 표로 작성하라.

$ \rm F = \dfrac{(202-32)}{(100-0)} C + 32 $

$ \rm F = \dfrac{9}{5} C + 32$

$ \rm C = \dfrac{5}{9} \left( F - 32 \right)$

>-40,	-40

>-30,	-22

>-20,	-4

>-10,	14

>0,	32

>10,	50

>20,	68

>30,	86

>40,	104



# 연습 문제 2


-2. 압력이 1000hPa 이고 온도가 10C 에서 수소 기체를 채집하였다. 비부피를 계산하라.

$ \rm p  \nu = R T $

$ \rm \nu = \dfrac{R T}{p} $

$ \rm \nu = \dfrac{R^{*} ~T}{m_{H_{2}}~p} $

$ \rm = \dfrac{8.314 \cdot 283.13}{2 \times 10^{-3} \cdot 10^{5}} = 11.77 ~ m^{3}~kg^{-1} $

$ \rm R^{*} = 8.3144 ~J ~mol^{-1} ~K^{-1}$

$ \rm m_{H_{2}} = 2 \times 10^{-3} ~kg~mol^{-1} $




# 연습 문제 3


-3. 건조공기를 구성하는 네가지 기체들에 대한 자료로부터 평균 분자량을 계산하라. 이 장에서 주어진 값과 여기서 구한 값을 비교하라.

성분, 화학식, 체적비(%), 분자량

질소	N2	78.084 28, 21.863

산소	O2	20.946	32 6.703

아르곤	Ar	0.934	40 0.374

이산화탄소	CO2	0.036	44 0.158

계산값 : 28.956, 교과서값 : 28.966


# 연습 문제 4


-4. 온도 200 K, 300 K, 400 K 에 대한 건조공기의 $ \rm \nu $, $ - p$ 다이아그램의 등온선을 계산하고 그려넣어라. 압력은 1000hPa 에서 200 hPa로 변동하며  $ \rm \nu $ 는 1 $\rm m^{3}~ kg{-1}$에서부터 2.5 $\rm m^{3}~ kg{-1}$ 까지 변동한다고 설정한다.

$ \rm p  \nu = R T $

$ \rm p = \dfrac{R T}{ \nu} $

$ \rm p = \dfrac{R^{*} ~T}{ \nu~ m} $



# 연습 문제 5


-5. 800 ~ 700 hPa 사이의 지오퍼텐셜 미터를 계산 하라. 두층 사이의 평균 온도는 -3C 이고 평균혼합비는 3 $ \rm g~kg^{-1}$ 이다.


$ \rm d \Psi  = g dz $

$ \rm dp = - \rho~g~dz $

$ \rm d \Psi  = - \nu~ dp $

습윤 공기의 경우

$ \rm d \Psi  = - R T^{*} \dfrac{dp}{p} $

$ \rm \Psi_2 - \Psi_1  =  R \int_{p_2}^{p_1} T^{*} \dfrac{dp}{p} $

$ \rm \Psi_2 - \Psi_1  =  -R \overline{ T^{*}} ln \dfrac{p_2}{p_1} $

m, s, K로 표시되는 단위를 가진 R에 9.8로 나누면

$ \rm \Psi_2 - \Psi_1  =  - \dfrac {R \overline{ T^{*}}}{9.8} ~ ln \dfrac{p_2}{p_1} $

$ \rm \Psi_2 - \Psi_1  =  - \dfrac {287.04 ~\cdot\overline{ T^{*}}}{9.8} ~ ln \dfrac{p_2}{p_1} $


$\rm T^{*} = T \left( 1- \dfrac{3}{8} \dfrac{e}{P} \right)^{-1}$


https://m.blog.naver.com/PostView.nhn?blogId=tnehf18&logNo=220423609713&proxyReferer=https%3A%2F%2Fwww.google.co.kr%2F

$\rm T^{*} = (1 + 0.61 q) T = (1 + 0.61 \dfrac{3}{1003}) \cdot 270 = 270.5$

$ \rm \Psi_2 - \Psi_1  =  - \dfrac {287.04 ~\cdot{ 270.5}}{9.8} ~ ln \dfrac{8}{7}  = 459.5$




# 연습 문제 6


-6. 절대온도는 아래 식에 의해서 지수함수로 냉각된다고 가정한다.

$ T = T_{0} e^{-\frac{z}{H}}$

여기서 $\rm T_{0} = 273~K$ 로서 z=0 에서의 온도이고, H는 $ T_{0}$인 등질대기의 높이이다. 아래 식이 됨을 보여라.

$ p = p_{0}~ e^{(1-e^{\frac{z}{H}})}$

여기서 $\rm p_{0}$는 z=0에서의 기압이다. 기온 감율이 건조단열감율과 일치하는 높이를 찾아라.



# 연습 문제 7

-7. 일정한 기온감율을 가진 대기 내에서 밀도가 높이에 따라 종속되는 수식을 유도하라.




# 연습 문제 8


-8. 공기 층의 밑면의 기압이 $p_1$이고 윗면의 기압이 $p_2$인 공기층을 생각하자. 만약 이 층 내의 가온도는 일정하다면 상층 기압변화의 증가분$dp_2$는 하층 기압의 변화 $dp_1$에 의해서 다음과 같이 주어짐을 보여라.

$ \dfrac{dp_1}{p_1} =  \dfrac{dp_2}{p_2}$




# 연습 문제 9


-9. 지상기압이 변동하지 않는 동안 시간에 따라 지상기온 T_0는 변도하나, 지상온도 T_0인 일정한 기온감율을 가진 대기를 고려하자. 어떤 고정된 고도에서 시간에 따라 기압의 변동율이 T_0에 상응하는 당질 대기의 높이에서 최대가 됨을 보여라.



# 연습 문제 10


-10. 어떤 관측소 기압계 고도는 해발 994 지오퍼텐셜미터이다. 최근 12시간 동안의 평균기온은 17.8C 였다. 완전히 보정한 관측소 기압은 890.0hPa이다. 공기의 기온감율을 건조단열감율인 6.5C km^{-1}으로 가정하여 해면 기압을 계산하라.

$p_{1} = p_{2} \dfrac{e^{\Psi_2 - \Psi_1}}{R~T^{*}}$

$ T^{*} = 273 + \dfrac{17.8 + 17.8 + 6.5 \times 0.994}{2} = 294.03$

$p_{1} = 890 ~\dfrac{e^{994}}{287.04~\cdot 294.03}$